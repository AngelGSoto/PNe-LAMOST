% mnras_template.tex 
%
% LaTeX template for creating an MNRAS paper
%
% v3.0 released 14 May 2015
% (version numbers match those of mnras.cls)
%
% Copyright (C) Royal Astronomical Society 2015
% Authors:
% Keith T. Smith (Royal Astronomical Society)

% Change log
%
% v3.0 May 2015
%    Renamed to match the new package name
%    Version number matches mnras.cls
%    A few minor tweaks to wording
% v1.0 September 2013
%    Beta testing only - never publicly released
%    First version: a simple (ish) template for creating an MNRAS paper

%%%%%%%%%%%%%%%%%%%%%%%%%%%%%%%%%%%%%%%%%%%%%%%%%%
% Basic setup. Most papers should leave these options alone.
\documentclass[fleqn,usenatbib]{mnras}

% MNRAS is set in Times font. If you don't have this installed (most LaTeX
% installations will be fine) or prefer the old Computer Modern fonts, comment
% out the following line
\usepackage{newtxtext,newtxmath}
% Depending on your LaTeX fonts installation, you might get better results with one of these:
%\usepackage{mathptmx}
%\usepackage{txfonts}

% Use vector fonts, so it zooms properly in on-screen viewing software
% Don't change these lines unless you know what you are doing
\usepackage[T1]{fontenc}

% Allow "Thomas van Noord" and "Simon de Laguarde" and alike to be sorted by "N" and "L" etc. in the bibliography.
% Write the name in the bibliography as "\VAN{Noord}{Van}{van} Noord, Thomas"
\DeclareRobustCommand{\VAN}[3]{#2}
\let\VANthebibliography\thebibliography
\def\thebibliography{\DeclareRobustCommand{\VAN}[3]{##3}\VANthebibliography}


%%%%% AUTHORS - PLACE YOUR OWN PACKAGES HERE %%%%%

% Only include extra packages if you really need them. Common packages are:
\usepackage{graphicx}	% Including figure files
\usepackage{amsmath}	% Advanced maths commands
% \usepackage{amssymb}	% Extra maths symbols

%%%%%%%%%%%%%%%%%%%%%%%%%%%%%%%%%%%%%%%%%%%%%%%%%%

%%%%% AUTHORS - PLACE YOUR OWN COMMANDS HERE %%%%%

% Please keep new commands to a minimum, and use \newcommand not \def to avoid
% overwriting existing commands. Example:
%\newcommand{\pcm}{\,cm$^{-2}$}	% per cm-squared

%%%%%%%%%%%%%%%%%%%%%%%%%%%%%%%%%%%%%%%%%%%%%%%%%%

%%%%%%%%%%%%%%%%%%% TITLE PAGE %%%%%%%%%%%%%%%%%%%

% Title of the paper, and the short title which is used in the headers.
% Keep the title short and informative.
\title[New high-ionization planetary nebula]{Discovery of a new high-ionization planetary nebula in LAMOST}

% The list of authors, and the short list which is used in the headers.
% If you need two or more lines of authors, add an extra line using \newauthor
\author[Guti\'errez-Soto et al.]{
  L. A. Guti\'errez-Soto$^{1,2}$
  \thanks{E-mail: gsoto.angel@gmail.com}
%A. N. Other,$^{2}$
%Third Author$^{2,3}$
%and Fourth Author$^{3}$
\\
% List of institutions
$^{1}$Instituto de Astrof\'{i}sica de La Plata (CCT La Plata - CONICET - UNLP), B1900FWA, La Plata, Argentina\\
$^{2}$Departamento de Astronomia, IAG, Universidade de S\~{a}o Paulo, Rua do Mat\~{a}o, 1226, 05509-900, S\~{a}o Paulo, Brazil\\
%$^{3}$Another Department, Different Institution, Street Address, City Postal Code, Country
}

% These dates will be filled out by the publisher
\date{Accepted XXX. Received YYY; in original form ZZZ}

% Enter the current year, for the copyright statements etc.
\pubyear{2015}

% Don't change these lines
\begin{document}
\label{firstpage}
\pagerange{\pageref{firstpage}--\pageref{lastpage}}
\maketitle

% Abstract of the paper
\begin{abstract}
  We report the identification of a new planetary nebula found in the catalog of
  emission line objects created by \citet{Skoda:2020} based in LAMOST data, using
  color-color diagrams based in GAIA and Pan-STARRS1 photometry. The optical LAMOST
  spectrum of this source show prominent emission lines, including highly ionized
  such as He II and [Ar V] lines. The LAMOST spectra was
  modelled using the code of photoionization  {\sc cloudy}.
  Comparing the model spectra with the observed one, we found
  the best-fit {\sc cloudy} models resulting in three best model fit. These model spectra have
   obtained a
  effective temperature of 14$\times10^{4}$ K. Other parameters such as luminosity
  and abundances were determined by comparing with the model.
  Comparing with evolution track of post-AGB I found that the mass
  of the progenitor star is 1.25 M{$\odot$}. 
\end{abstract}

% Select between one and six entries from the list of approved keywords.
% Don't make up new ones.
\begin{keywords}
planetary nebulae: general -- ISM: lines and bands -- surveys
\end{keywords}

%%%%%%%%%%%%%%%%%%%%%%%%%%%%%%%%%%%%%%%%%%%%%%%%%%

%%%%%%%%%%%%%%%%% BODY OF PAPER %%%%%%%%%%%%%%%%%%

\section{Introduction}

\label{sec:intro}

Planetary nebulae (PNe) represent the last stage of evolution of low- and intermediate-mass stars
(0.8$\msol$ - 8.0$\msol$). The PNe phase begin when gas is ejected from the red giant stars late
in their lives and subsequently this gas is ionized by the radiation field coming from the remnant
star resulted. An emission nebula expands, a glowing shell of ionized gas until lost in the
interstellar medium. Then the dying star core becomes a white dwarf.

The number of the PNe discovered in the galaxy is relatively low (\(\sim 3,500\)). However,
this current number of PNe is far away (which represents only about 15-30\%) of the estimated
total of Galactic PNe (\citealp{Frew:2008}; \citealp{Jacoby:2010}) showing that a small fraction
of the PNe have been cataloged \citep{Frew:2017}. If this is true, there will be a much larger
number of PNe in the Galaxy than if a special condition
is required. The models of \citet{Moe:2006}, for
example, predict that there are 46,000 $\pm$ 22,000 for the general case,
but only $\sim$6600 \citep{Marco:2005} if
close binaries (e.g. common-envelope phase) are required.
\citet{Miszalski:2009} confirm earlier estimates that the
binary fraction of PN central stars is only 10-20\% of all
PNe, and thus, binarity is not likely to be a major factor in
the formation process. If true, there should be many PNe
waiting to be found in the Galaxy. This mean that the search for planetary nebulae
becoming a important task, that eavery time is more dificult, due to many of
the undiscovered PNe are probably
the more distant and then the more the weak. And because many of them, probably, are located
in nuvens of dust. And note that planetary nebulae only last for about 5,000-
25,000 yr \citep{Badenes:2015}, making them a very short-lived part of the stellar life cycle.
But is import to discovery new planetary nebula? the answer could be very obvious and is
related with the idea that PNe provide vital clues for the understanding late-stage stellar
evolution and Galactic chemical enrichment. Their strong emission lines allow the determination
of abundances, expansion and radial velocities, and
CSPN temperatures. The PN phase
enriches the ISM with nitrogen, carbon, helium, and dust,
important components in the formation of future generations
of stars. PNe yield information on the nuclear
burning, dredge up, and mass loss in the stellar progenitor
(see \citealp{Kwitter:2022} for an excellent recent PN
review). Thus, we need to know the number of PNe in
the Galaxy in order to develop accurate models of the
chemical enrichment rates since the initial burst of Type II
supernovae when the Galaxy was young.

PN studies have been hampered by three problems:
(i) the previous lack of accurate distances to most Galactic
PNe; (ii) obtaining representative PNe samples of the true
population diversity \citep{Parker:2022}, and (iii) their unknown
progenitor masses. The first problem has prospects of
resolution via accurate Gaia CSPN distances, though many
CSPNe remain too distant and faint for Gaia DR3 and correct
CSPN identification remains an issue for some \citep{Parker:2022}.
The second problem is being addressed by deep, narrow-band,
wide-field surveys, e.g., \citet{Parker:2005},
\citet{Drew:2005}, and \citet{Drew:2014}. For the third problem
of progenitor masses, these can only
be accurately determined for PNe in Galactic globular and
open clusters (OCs). These allow precise distance determinations
from color-magnitude diagrams (CMD) and Gaia \citep{Fragkou:2022}.
Then, these problems could be addressed by increasing the
number of PNe in the Galaxy, because a better study and statistical
could been done. For this reason, several attempted have been performed in the past
based diagnostic diagrams using several
emission-line intensities to differentiate resolved emission
sources such as PN from H {\sc ii} regions, SNRs and other objects.
Diagnostic diagram based on the H{$\alpha$}/[N II] and H{$\alpha$}/[S II]
emission-line ratios \citep{Sabbadin:1977, Fesen:1985, Riesgo:2006}.
The selection criteria were empirically derived in theses diagrams,
by using emission lines ratios observed in supernova remnants, PNe and H II regions.
\citet{Viironen:2009a, Viironen:2009b} used two color-color diagrams based on IPHAS
and 2MASS photometry to discriminate PNe from stellar and other emission lines sources
in the Galactic context. 

Recently, \citet{Akras:2019b} using the classification tree model
found new colour criteria to identify compact PNe  by
using both 2MASS and AllWISE photometric data. \citet{Gutierrez-Soto:2020}
searched for compact PNe in the Javalambre and Southern Photometric Local
Universe Survey data (J-PLUS and S-PLUS, respectively), using a combination
of narrow- and broad-band photometry. And new diagnostic diagram were proposed
by \citet{Vejar:2019} to find for new PNe using broad-band filters.
In this order of ideas, in an attempt to create new color-color criteria based
in the larges photometric surveys: GAIA and Pan-STARRS, we find a color criteria
based on theses surveys, on which PNe with strong H{$\alpha$} are puts in
evidence. We applied this criterion to the sample of emission lines constructed
under the LAMOST data. One PNe nebula was discovery, J020808.63+491401.0,.
At the moment of this manuscript was written I perceived that the
object now appear classified as galaxy in SIMBAD, but our analysis
suggested that the source is in reality an high-excitation PN. 

\section{Surveys}
\label{sec:surveys}

In order to bring to the reader a small overview of the data used
in this manuscript, here I present in a few words the three
surveys basis of this work.

\subsection{GAIA EDR3}
\label{sec:gaia}

The early installment of the third
GAIA data release (EDR3; \citealp{Brown:2021}) was made available to the public
on 2020 December, which contain which contains the first 34 months
of data collection of the Gaia mission \citep{Brown:2018}.
It consist of around 1.8 billion sources
with precise astrometry as well as photometry in
the $G$, $G_{BP}$ and $G_{RP}$ bands. aia EDR3 made
significant improvements in precision and accuracy of photo-
metric magnitudes, parallax, celestial positions, and proper
motions with respect to Gaia DR2. This means that it improved
the efficiency of candidate selection of a variety of
objects (SOME REFERENCE HERE).

\subsection{Pan-STARRS DR1}
\label{sec:PS1}

Pan-STARRS1 (PS1)
is a multi-wavelength, multi-epoch, optical imaging survey
which covers $\approx$75\% of the sky. PS1 goes $\sim$1 mag fainter
than SDSS in the z band (York et al. 2000). The survey’s five optical
filters, gP1, rP1, iP1, zP1, and yP1, are described in Stubbs et al.
(2010) and Tonry et al. (2012). At each epoch a single field
is exposed for 60 s in gP1, 38 s in rP1, and 30 s in iP1, zP1,
and yP1. The photometry and astrometry from each epoch
have been combined to obtain average magnitudes and proper
motions.

\subsection{LAMOST}
\label{sec:lamost}

The Large Sky Area Multi-Object Fiber Spectroscopic Telescope
(LAMOST, also named Guoshoujing Telescope) is a
telescope dedicated for spectroscopic sky survey.
There are 4000 fibers within a diameter of 1.75 meters (corresponding
to 5$^{\circ}$ in the sky) at the focal surface \citep{Cui:2012}.
The Low-Resolution Spectroscopic Survey (LRS) of LAMOST
began in October 2011, with a spectral resolving power ($R = \lambda/\Delta\lambda$)
of about 1800 and a wavelength coverage of
3700-9100~\AA \citep{Zhao:2012}. The first year observation was
for the pilot survey \citep{Luo:2012, Zhao:2012},
and the regular survey began in September 2012. And since then
a larger amount  spectroscopic data are been produced for
scientific propose.
Each spectrum
is spliced into the blue part and the red part. The integration
time is 600, 1500, or 1800 s. All of the low-resolution spectra
were generally classified into different types through the
pipeline, including 9,231,057 stellar spectra, 177,270 galaxy
spectra, 62,168 quasar spectra, and 440,842 spectra marked as
unknown. The massive spectral database obtained by
LAMOST is a rich resource for us to identify different kinds
of peculiar stars (Yongkang Sun et al. 2021).

\section{Methodology}
\label{sec:metho}

At the beginning the idea was to identify for new planetary nebula in GAIA.
I started to constructed possibles color-color diagrams to separate PNe from
other emission lines objects and stars using GAIA only. The separation was not
good because PNe, normal stars and other emission line stars like CVs occupy the same region
in the diagram. Then, I decided to move to Pan-STARRS and combining the two
surveys; Pan-STARRS1 and GAIA EDR3. I found one color diagram that isolate the PNe
with, I suppose, strong \ha{} emission line. By using the \((G - r)\) versus \((G_{BP} - G_{RP})\)
color-color is possible to separate those PNe excess in the in the $r$-broad-band
filter as is possible to see in the Fig~\ref{fig:gaia-ps}. This plot include cataclysm variables
\citep{Downes:2006}, supernova remnants \citep{Green:2019}, symbiotic starts from \citet{Akras:2019a}.  
Fig~\ref{fig:gaia-ps}
shows that the many of the PNe have value in the color \(G - r\) = 0, however several
PNe cover the interval in this same color between 0 and 8, and span between -1 and 5.
The orange contours represent other emission line objects that includes CV, SySt,
YSOs, AeBe stars and SNRs. The contours becoming white at the outside indicating
that the number of objects decrease significantly. And the blue contours signify
stars from \citet{Smart:2021}. Where they occupy the region with \((G - r)\) = 0.

\begin{figure}
\centering
  \includegraphics[width=0.9\linewidth]{Figs/color-diagram-ps-gaiaEDR3.pdf}
  \caption{} 
  \label{fig:gaia-ps}
\end{figure}

All indicate that the PNe with strong \ha{} can be selected with this color criteria??? What indicate the $G$-magnitude?.
Then, to see the possibility of using this color criterion to select these PNe, by
testing and using the emission line catalog from  \citet{Skoda:2020}. First, they
identify emission line objects from LAMOST  implementing a machine learning approach.
Then, they divided their final sample of emission line objects in tree subgroups. Those with \texttt{SIMBAD}
coincidences, those that are listed by \citet{Hou:2016} and another list that are neither cross-matched with
SIMBAD nor listed by  \citet{Hou:2016}. To texting the possibility to find for new PNe with the color criteria
explained above, we applied it directly in the new list, which present objects not reported
previously in the literature, which is a list with 1000 objects.

\begin{figure}
\centering
  \includegraphics[width=0.9\linewidth]{Figs/pn-candidates-gaiaDR3.pdf}
  \caption{} 
  \label{fig:gaia-ps-apply}
\end{figure}

Four objects met this  condition as is possible to see in the Fig~\ref{fig:gaia-ps-apply}.
We downloaded the low resolution spectra of these objects from the LAMOST database. 
Tree objects of them display strong \ha{}, but are not display the other emission
lines typically of PNe like [O III], He II, [S II], among other. But the three look likes as PNe.
Because, it displays He II emission lines, the Balmer ones, [O III], among others.
Fig~\ref{fig:spectra} the spectrum of the new PN finding in the list of emission line of
\citet{Skoda:2020}. This PNe seem a very high ionization object, by eye is possible to see
that the He II emission line is as strong as the H$\beta$ line. No lines of ions in low stages
of ionization were detected, lines as [N II] and [O II].


Given the LAMOST spectra are not calibrated in flux but they are flux relative,
then I think that some ratio line can be calculated. Could be?

\begin{figure*}
\centering
  \includegraphics[width=\linewidth]{Figs/spec-56581-VB031N50V1_sp08-218.pdf}
  \caption{Low resolution spectrum from LAMOST of J020808.63+491401.0.
    The most prominent emission lines detected in the spectra are given by the dashed vertical
    lines.} 
  \label{fig:spectra}
\end{figure*}

The images of the PN called J020808.63+491401.0 in LAMOST are showed
in Fig~\ref{fig:image}.
\textit{Left panel} exhibits the PanSTARRS coloured
images \footnote{These RGB images were made by implementing
the python package \texttt{aplpy} \citep{aplpy:2019}}, which
was performed by combining the $g$, $r$ and $i$ filters in
the blue, green and red colour channels, respectively.
The image shows clearly a nebular component surrounding 
a central star. \textit{Right panel} shows the
WISE RGB image, with the filter W1, W2, and W4 in
the blue, green and red channels, respectively.
 The WISE image shows that the object is W4 wright.  
At 22$\mu$ (W4-filter) it appears as an almost
circular (but slightly elongated in the north-south direction)
diffuse halo (of angular diameter of $\simeq$ 50 arcsec) surrounding
a core of bright emission centred around J020808.63+491401.0. 


\begin{figure*}
  \centering
  \begin{tabular}{l l}
\includegraphics[width=0.5\linewidth]{Figs/cutout_rings_v3_skycell_2294_031_stk_i_unconv-irg-RGB.pdf}
\includegraphics[width=0.5\linewidth]{Figs/w4_ra32_035994_dec49_233615-421-RGB.pdf}
\end{tabular}  
  \caption{Pan-STARRS optical (\textit{left}) and WISE IR \textit{right} coloured
    images of the new PN. To create the optical coloured image were used the $g$, $r$ and $i$
    images for the blue, green and red channels, respectively. In the say way, w1, w2 and w4
    filters were used to create the IR-image.} 
  \label{fig:image}
\end{figure*}


\section{Comparing with other high-ionization PNe}
\label{sec:comp}

In order to check if our sample consists of high ionization PNe,
we will compare the radio and infrared properties of the
argets belonging to our sample with results obtained in
similar studies. In particular, in the following, we will
consider the work of Aaquist & Kwok (1991, AK91) carried
out with the VLA at 15 GHz on a sample of young PNe
selected on the basis of their compact radio morphology.
The targets of AK91 were also observed at 5 GHz, but
we prefer to compare our results with those obtained at
15 GHz since in both cases optical depth effects should
not be important.
All the selected targets in AK91 have high brightness tem-
peratures, infrared excess (IRE) much higher than unity
and dust temperature higher than the typical value ob-
served in more evolved nebulae, which is of the order of
100 K (Pottasch et al., 1984). All these properties are
consistent with the hypothesis that the sample consists of
very young PNe.

In order to compare the physical properties of the
nebulae belonging to AK91 with our sample we plot in
Figs. 1, 2 and 3 the brightness temperature (TB), the
emission measure (EM) and the infrared excess (IRE) of
both samples. Those quantities were re-calculated from
the published radio measurements using the same formu-
las as for our sample.
It is evident that for the AK91 sample TB and EM
are systematically higher than for our sample, and this
seems to indicate that our sample indeed consists of more
evolved PNe.
On the contrary, the infrared excess of our sample, which
has values systematically higher than those reported by
AK91, indicates a particularly young sample of PNe. This
apparent contradiction is further complicated by the fact
that the infrared properties of both samples are quite sim-
ilar, as is evident from an inspection of the IRAS color-
color diagram (Fig. 4), where sources belonging to dif-
ferent samples occupy the same region of the diagram.
This region is also shared with SAO244567, the younges
known PN, whose evolution appears to be quite rapid,
since it has become ionized only within the past 20 years
(Parthasarathy et al., 1993). {\cs A search for very young
  Planetary Nebulae (Umana et al. 2013)

\begin{figure*}
\centering
\includegraphics[width=\linewidth]{Figs/spectra-compare.pdf}
\caption{Spectra of 3 known PNe and the new PN. From upper to lower the ID of the
  sources are PN PRTM 1, NGC 4361, NGC 2242, and the recent discovery J020808.63+491401.0.
  The spectra have all been scaled and normalized for display purposes.} 
  \label{fig:compare-spectra}
\end{figure*}

In Fig~\ref{fig:compare-spectra} we are compare our PNe with
other high-ionization PNe (NGC 2242, NGC 4361 and PRTM 1).
Note that all three PNe are objects located at high latitudes,
this means that they belong to the halo Galactic.
All four spectra are very similar with the same Balmer lines,
the high-ionization lines and lacked the low-ionization lines.

\begin{figure}
  \centering
  \begin{tabular}{l l}
\includegraphics[width=0.52\linewidth]{Figs/cutout_rings.v3.skycell.2243.029.stk.i.unconv-irg-RGB.pdf}
\includegraphics[width=0.5\linewidth]{Figs/0979p454_ac51-w4-int-3_ra98.53061791727998_dec44.77716333248_asec200.000-421-RGB.pdf}\\
\includegraphics[width=0.52\linewidth]{Figs/cutout_rings.v3.skycell.0924.030.stk.i.unconv-irg-RGB.pdf}
\includegraphics[width=0.5\linewidth]{Figs/1855m182_ac51-w4-int-3_ra186.12812938647002_dec-18.78487981564_asec200.000-421-RGB}\\
\includegraphics[width=0.535\linewidth, trim=280 10 330 10, clip]{Figs/dss_search_red.pdf}
\includegraphics[width=0.47\linewidth, trim=58 0 0 0]{Figs/0754m394_ac51-w4-int-3_ra75.75721626934_dec-39.76236833917_asec150.000-421-RGB.pdf}\\

\end{tabular}  
  \caption{Optical (\textit{right}) and IR (\textit{left}) images of the other high-ionization PNe.. For comparison. } 
  \label{fig:images-known}
\end{figure}

Fig~\ref{fig:images-known} show the optical and IR images of the known planetary nebulae
presented in section ?? apparently all this PN have the same shape, e.g. are around like
the Lamost PN. Also in all them the central star is clearly perceptible like the Lamost PN,
indicating, probably, the high temperature nature on their central star.

\subsection{Why is not a supernova?}
\label{sec:snr}

Given supernovas are also high-ionization sources the object could be
a supernova, but the spectra don show broad lines as the classical supernova.

Two types of galactic objects show such high excitation
lines: dusty planetary nebulae (PNe, Bernard-Salas et al. 2009;
Stanghellini et al. 2007; Guiles et al. 2007) and supernova rem-
nants (SNRs, Sandstrom et al. 2009; Ghavamian et al. 2009).
SNRs typically show very broad emission lines, produced by the
high velocity shock waves (Fesen et al. 1985; Fesen & Hurford
1996; Stupar et al. 2007). PNe, on the other hand, are character-
ized by narrow emission lines arising from the low velocity ex-
panding outer shells (Balick & Frank 2002; Górny et al. 2009).
Both classes of objects have been extensively studied by several
authors, although just a few are so dusty that they were initially
discovered only at mid-infrared wavelengths. To distinguish be-
tween these two possibilities, further spectroscopy on OL17 was
needed \citep{Oliveira:2011}.

\section{Modeling the observed spectra}
\label{sec:model}

% Example table
\begin{table}
	\centering
	\caption{Best-fit {\sc cloudy} model parameters for LAMOST J020808.63+491401.0.}
	\label{tab:example_table}
	\begin{tabular}{lcccc} % four columns, alignment for each
                \hline
                & Model 1 &  Model 2 &  Model 3\\  
		\hline
		Parameter & & Value \\
                \hline
		$\log(T_{\mathrm{BB}}) (\mathrm{K})$  & 5.146 \\
		$\log(\mathrm{luminosity) (erg~s^{-1})}$ & 37.571 \\
		  $\log(\mathrm{Hden) (cm^{-3})} $ & 3.477 \\
                  $\log(\mathrm{R_{in}) (cm)}$ & 16.903 \\
                $\log(\mathrm{R_{out}) (cm)}$ & 17.350 \\
                 Distance (pc) & 2313.01 \\
                  $\log(\mathrm{He/H})$ & -0.80 \\
                $\log(\mathrm{C/H})$ & -4.15 \\
                $\log(\mathrm{N/H})$ & -4.72 \\
                $\log(\mathrm{O/H})$ & -3.81 \\
                $\log(\mathrm{Ne/H})$ & -4.58 \\
                $\log(\mathrm{Si/H})$ & -5.00 \\
                $\log(\mathrm{S/H})$ & -6.00 \\
                $\log(\mathrm{Ar/H})$ & -6.24 \\
                 \hline
                 $\chi^2$ & 6.78 &9.55& 9.67& \\
                 $\chi^2_r$ & 1.13 & 1.59&1.61& \\
                 
                \hline
	\end{tabular}
\end{table}

Given the small number of observational constraints, readopt a very
simple model of a planetary nebula which
consists of a homogeneous gaseous sphere surrounding ahot star radiating as a blackbod.
Despite the LAMOST spectra are not in physical flux unity, they have flux relative.
This means itis possible compare it with other spectra in physical units.
For that, I think, it is pertinent to
compare the LAMOST spectra of our PN with models.

The observed spectrum was modelled by using the {\sc cloudy}
photo-ionization code version c22.01 \citep{Ferland:2017}. This
code is based on detailed microphysics to simulate the physical
conditions of nonequilibrium gas clouds exposed to an external
radiation field. It solves the thermal, statistical, and chemical
equilibrium equations self-consistently. Currently, it uses 625
species including atoms, ions, and molecules and five distinct
databases: H-like and He-like isoelectronic sequences \citep{Porter:2012},
Stout \citep{Lykins:2015}, CHIANTI \citep{Landi:2012},
LAMDA \citep{Schoier:2005}, and the H2 molecule
\citep{Shaw:2005} to model the spectral lines. All the known
important ionization processes, e.g., photo, Auger, collisional
and charge transfer and recombination process, namely,
radiative, dielectronic, three-body recombination, and charge
transfer are included self-consistently. {\sc cloudy} predicts both
the intensities and column densities of a very large number
($\sim 10^{4}$) of spectral lines covering the whole electromagnetic
range, from non-local thermodynamic equilibrium (NLTE),
illuminated gas clouds by solving the equations of thermal and
statistical equilibrium for a given set of input parameters. More
details about {\sc cloudy} could be found in \citet{Ferland:2013}
and \citet{Pandey:2022}. The code uses a set of input parameters to compute the
ionization, thermal, and chemical state of a non-equilibrium gas cloud,
illuminated by a central source and predict the resulting spectra. 
In the past, \citet{Vejar:2019} used {\sc cloudy} to produce
synthetic spectra of PNe to simulate the  Broadband photometry
of Large Synoptic Survey Telescope(LSST). In the same way, a grid of modelled
halo galactic PNe were performed to simulate the J-PLUS and S-PLUS photometry
to developed new color criteria to find for PNe in these surveys
by \citet{Gutierrez-Soto:2020}. We used {\sc pyCloudy} \citep{Morisset:2013} a
package python\footnote{\url{https://sites.google.com/site/pycloudy/home}} which
is a set of tools to deal with photoionization code {\sc cloudy} (\url{www.nublado.org}),
which was used to create a set of inputs for the model and deal with the output files.

We consider a central ionizing source surrounded by a spherically symmetric
gaseous material on which dimensions are determined by the inner ($R_{\mathrm{in}}$) and
outer ($R_{\mathrm{out}}$) radii (cm).The central ionizing radiation is assumed to have a
black-body shape with temperature T$_{\mathrm{BB}}$ (K) and luminosity L
(ergs$^-1$). We assume spherical geometry for the nebula, a uniform filling factor of unity.
Many models were generated with the intention of finding the best fit.
In agreement with the LAMOST spectra and similarity with theses high-ionization PNe
(see \ref{fig:compare-spectra})
we considered a set of effective temperature between 10$\times10^4$ and 20$\times10^4$
in step of 10$\times10^3$ and luminosities from 400$L_{\odot}$ to 10100$L_{\odot}$ in steps
of 300$L_{\odot}$. We also considered several hydrogen density (500, 1000, 3000 and 6000cm$^{-3}$).
(We varied TBB in the
range of 104 - 106 K, L in the range of 1036 - 1038 erg s-1, n(H)
in the range of 108 - 1012 $cm^{-3}$, simultaneously with the elemental abundances.)
We note that the spectra of LAMOST J020808.63+491401.0 is particularly quiet similar to the
spectrum of NGC2242, so we first adopted the abundances of it and after we change the
value of He and Ar to get a better match with the observed spectra. We also adopted inner
and outer radius
of that showed very good. For the distance we implemented the distance of the central star
of the PNe estimated by \citet{Bailer:2021} based on
the parallax of GAIA of 2.313 kpc. Note that this is the geometrical distances estimated for them. 
Fig~\ref{fig:spectra-obs-model} shows a comparison
between LAMOST spectra and a {\sc cloudy}. This model was taken from a grid of
models that reproduce spectra form the Galactic halo. For this, it don't exist a real matches between the two spectra.
I will a perform a grid of PN models that better reproduce the LAMOST spectra of our candidate.
However, almost all the line are reproduce with this model, that could be interesting, because the
temperature effective of this model is 130$\times10^3$K, indicating a very high excitation object.
Maybe, it will be good idea estimate ratio lines to confirm that! is it possible with this spectrum?
The observed spectrum and the modeled spectrum were
matched using fluxes relative to the H{$\beta$} emission line.

\subsection{The best fit model}
\label{sec:best-fit}

\citet{Helton:2010, Mondal:2018, Pavana:2019, Mondal:2020, Pandey:2022a, Pandey:2022b}
implemented the $\chi^2$ and subsequently $\chi^2_r$ to compare observed spectroscopic data,
mainly novas sources, with modelled {\sc cloudy} spectra to find the
fit models.
Following these authors, I chose the final models
after many iterations of multiple test models based on
varying the input parameter show before. For that
I compare the model-generated lines with the fluxes of the observed lines.
This allows determined automatically the goodness of fit by
calculating $\chi^{2}$
of the models given by the following relation,

\begin{equation}
  \chi^{2} = \sum^{n}_{i = 1} \frac{(M_i - O_i)^2}{\sigma^{2}_i}
  \label{eq:chi}
\end{equation}

where $M_i$, $O_i$ and $\sigma^{2}_i$ are the modelled line flux ratios, the observed line flux
ratios and uncertainty in the observed line flux ratios, respectively. I also used the $\chi^{2}$
to find the best-fitting model, which is given by,

\begin{equation}
  \chi^{2}_{r} = \frac{\chi^{2}}{\nu}
  \label{eq:chi-red}
\end{equation}

where $\nu$ is represents the number of degrees of freedom (DOF), which is estimate
by implementing $\nu = n - n_p$, with $n$ and $n_p$ being the number of observed lines
used to estimate $\chi^{2}$ and the number of parameters, respectively. Note that for a reasonably
fit, the value of $\chi^{2} \sim \nu$. This means that the value of $\chi^{2}_r$ should be low,
usually between 1 and 2. 

The flux lines of each line were measured  interactively by fitting a 1D Gaussian using the 10 $\AA$ centered on
the line and integrate over this best fit Gaussian within more or less 3$\sigma$ interval. Being $\sigma$ the
standard deviation of the individual line fit Gaussian. I found automatically the
parameters to fit the best 1D Gaussian model using the function \texttt{estimate\_line\_parameters} from \texttt{Astropy specutils.fitting}
package \footnote{\url{https://specutils.readthedocs.io/en/stable/index.html}}.

The uncertainty in the flux ratios of the observed lines were computed following the
equation presented by \citet{Tresse:1999}:

\begin{equation}
  \sigma_{F} = \sigma_{c} D \sqrt{2N_{\text{pix}} + \frac{\text{EW}}{D}
  \label{eq:err-sigma}
\end{equation}

respectively, where $\sigma_{c}$ is the mean standard deviation per pixel of
the continuum on each side of the line, $N_{\text{pix}}$ is the number of pixels
under the line and $D = 1.0009 \AA$ pixel$^{-1}$ is the spectral dispersion.

% Example table
\begin{table*}
	\centering
	\caption{Observed and best-fit {\sc cloudy}  model line fluxes for LAMOST J020808.63+491401.0.}
	\label{tab:abundances}
	\begin{tabular}{lcccccccc} % four columns, alignment for each
                \hline
		\hline
		Line & $\lambda$(\AA) & Observed Flux  & Model 1 Flux & $\chi^{2}$ & Model 2 Flux &  $\chi^{2}$ & Model 3 Flux &  $\chi^{2}$ \\
		\hline
		[Ne III] + H7  & 5.146 \\
		H{$\delta$} & 37.571 \\
		H{$\gamma$}  & 3.477 \\
                He II & 16.903 \\
                H{$\beta$} & 17.350 \\
                $[\text{O III}]$ & 2313.01 \\
                $[\text{O III}]$ & -0.80 \\
                $[\text{Fe III}]$ & -4.15 \\
                H{$\alpha$} & -4.72 
                \hline
	\end{tabular}
\end{table*}

The low $\chi^{2}_{r}$
 (< 2) values of the model in the present study indicates
that the {\sc cloudy} model generated spectra match well with
the observed spectra. Even while our model reproduces a wide variety
of observable effects, the phenomenology has certain limits.
In this sense, from the 10000 models generated only three models
match very wells with the spectra of  LAMOST J020808.63+491401.0,
the values of $\chi^{2}$ and $\chi^{2}_{r}$ of the three best-fit spectra
are shown in the table~\ref{tab:abundances}. The three
models have values of $\chi^{2}_{r}$ between 1 and 2, satisfying
the criteria for an acceptable fit. Being the best model-fit ($\chi^{2}_{r} = 1.2$)
those which is the  planetary nebula spectrum model with the
most high effective temperature (T$_{\text{eff}}$ = 160$\times$10$^{4}$).

\begin{figure*}
\centering
\includegraphics[width=0.89\linewidth, trim=10 90 10 10, clip]{Figs/model_140000_37.15_3.70.pdf}
\includegraphics[width=0.89\linewidth, trim=10 90 10 10, clip]{Figs/model_150000_36.98_3.60.pdf}
\includegraphics[width=0.89\linewidth, trim=10 10 10 10, clip]{Figs/model_140000_37.25_3.78.pdf}
\caption{The three best fitted models, on which the $\chi^2_r$ value are 1.13, 1.59, 1.61.
  The spectra were normalised to H{$\beta$}} 
  \label{fig:spectra-obs-model}
\end{figure*}

\section{Comparison with post-AGB stellar evolution tracks}
\label{sec:tracks}

\begin{figure}
\centering
  \includegraphics[width=\linewidth]{Figs/hr-planetarieNebula}
  \caption{HR diagram of PN central stars, showing the observed luminosity
and effective temperature of the new PN, compared with
post-AGB evolutionary tracks from \citet{Miller:2016} (solid lines,
labelled with initial stellar mass in solar masses). Small star symbols show
an evolutionary time along each track equal to the kinematic age of the
intermediate shell of NGC 2242, with blue shading indicating 50 per cent
variation about this value.  Filled symbols shows the known PN showed in section~\ref{}
that have been selected to be  similar to the new PNe.} 
 \label{fig:track-evolutive}
\end{figure}

Fig~\ref{fig:track-evolutive} shows the position of the new PN (blue stars) in the luminosity
versus effective temperature diagram. The position of the three PNe on which we have used to
compare with the new one are also shown (orange circles). Post-AGB stellar evolutionary tracks for
approximately solar metallicity \citep{Miller:2016} are indicated for the continue lines.
I mean for stars with M = 1, 1.25, 1.5, 2.0, 2.5 and 3M$_{\odot}$. In agreement with the position of LAMOST J020808.63+491401.0 in the diagram the mas of the progenitor star is between 2.5 and 3.0M$_{\odot}$.
 d
\section{Conclusions}
\label{sec:conclu}

In this manusncrpt, I want to report a new planatary nebula found
in the catalog of emission line spectra of \citet{Skoda:2020}.
The PN was found by applying color criteria based on panstarr
and GAIA to the sample of emission objects.
Given that the \citet{Skoda:2020} sample was constructed
with the LAMOST data, there is spectra of the objects selected
allowing confirmed the PN nature or not.
Bla bla bla bla bla bla

\section*{Acknowledgements}

LAG-S acknowledges funding for this work
from CONICET and FAPESP grants 2019/26412-0.
The Pan-STARRS1 (PS1) Surveys and the PS1 public science
archive have been made possible through contributions by the
Institute for Astronomy, the University of Hawaii, the Pan-
STARRS Project Office, the Max Planck Society and its
participating institutes, the Max Planck Institute for Astronomy,
Heidelberg, and the Max Planck Institute for Extraterrestrial
Physics, Garching, the Johns Hopkins University, Durham
University, the University of Edinburgh, the Queen’s University
Belfast, the Harvard-Smithsonian Center for Astrophysics, the
Las Cumbres Observatory Global Telescope Network Incorpo-
rated, the National Central University of Taiwan, the Space
Telescope Science Institute, the National Aeronautics and Space
Administration under grant No. NNX08AR22G issued through
the Planetary Science Division of the NASA Science Mission
Directorate, National Science Foundation grant No. AST-1238877,
the University of Maryland, Eotvos Lorand University
(ELTE), the Los Alamos National Laboratory, and the Gordon
and Betty Moore Foundation.
This work presents results from the European Space Agency
(ESA) space mission Gaia. Gaia data are being processed by
the Gaia Data Processing and Analysis Consortium (DPAC).
Funding for the DPAC is provided by national institutions, in
particular the institutions participating in the Gaia MultiLat-
eral Agreement (MLA). The Gaia mission website is \url{https:
//www.cosmos.esa.int/gaia}. The Gaia archive website is
\url{https://archives.esac.esa.int/gaia}.
Guoshoujing Telescope (the Large Sky Area Multi-Object Fiber Spectroscopic
Telescope LAMOST) is a National Major Scientific Project built by the Chinese
Academy of Sciences. Funding for the project has been provided by the National
Development and Reform Commission. LAMOST is operated and managed by the
National Astronomical Observatories, Chinese Academy of Sciences.
Scientific software
and databases used in this work include 
TOPCAT\footnote{\url{http://www.star.bristol.ac.uk/~mbt/topcat/}} \citep{Taylor:2005}, 
simbad and vizier from Strasbourg Astronomical Data Center (CDS)\footnote{\url{https://cds.u-strasbg.fr/}} 
and the following  python packages: numpy, astropy, specutils, APLpy, matplotlib, seaborn.
%%%%%%%%%%%%%%%%%%%%%%%%%%%%%%%%%%%%%%%%%%%%%%%%%%
\section*{Data Availability}

The inclusion of a Data Availability Statement is a requirement for articles published in MNRAS.
Data Availability Statements provide a standardised format for readers to understand the availability
of data underlying the research results described in the article. The statement may refer to original
data generated in the course of the study or to third-party data analysed in the article.
The statement should describe and provide means of access, where possible, by linking to the
data or providing the required accession numbers for the relevant databases or DOIs.


%%%%%%%%%%%%%%%%%%%% REFERENCES %%%%%%%%%%%%%%%%%%

% The best way to enter references is to use BibTeX:

\bibliographystyle{mnras}
\bibliography{Ref-pne} % if your bibtex file is called example.bib


% Alternatively you could enter them by hand, like this:
% This method is tedious and prone to error if you have lots of references
%\begin{thebibliography}{99}
%\bibitem[\protect\citeauthoryear{Author}{2012}]{Author2012}
%Author A.~N., 2013, Journal of Improbable Astronomy, 1, 1
%\bibitem[\protect\citeauthoryear{Others}{2013}]{Others2013}
%Others S., 2012, Journal of Interesting Stuff, 17, 198
%\end{thebibliography}

%%%%%%%%%%%%%%%%%%%%%%%%%%%%%%%%%%%%%%%%%%%%%%%%%%

%%%%%%%%%%%%%%%%% APPENDICES %%%%%%%%%%%%%%%%%%%%%

\appendix
\section{1D-Gaussian fitted to the emission lines}
The 1D-Gaussian fitted to the emission lines of J020808.63+491401.0 used to measurement the flux lines and subsequently used to estimate the $\chi^2$ to
find the best-fitted {\sc cloudy} models.
\begin{table*}
\centering
  \caption{The best fitted models, on which the $\chi^{2}_r$ have value between 1 and 2. \label{tab:best-model12}}\
  \begin{tabular}{l l l }
  \includegraphics[width=0.3\linewidth, clip]{Figs/Obs_[NeIII]+H7.pdf} & \includegraphics[width=0.3\linewidth, clip]{Figs/Obs_Hdelta.pdf} & \includegraphics[width=0.3\linewidth, clip]{Figs/Obs_Hgamma.pdf} \\ \includegraphics[width=0.3\linewidth, clip]{Figs/Obs_HeII.pdf} &
    \includegraphics[width=0.3\linewidth, clip]{Figs/Obs_Hbeta.pdf} & \includegraphics[width=0.3\linewidth, clip]{Figs/Obs_[OIII]4958.pdf} \\ \includegraphics[width=0.3\linewidth, clip]{Figs/Obs_[OIII]5006.pdf} & \includegraphics[width=0.3\linewidth, clip]{Figs/Obs_[FeIII].pdf} &
    \includegraphics[width=0.3\linewidth, clip]{Figs/Obs_Halpha.pdf} \\
\end{tabular}
\end{table*}

\section{The best fitted models}
The Fig show the {\sc cloudy} models with value on the $\chi^{2}_r$ between 1 and 2.
\begin{table*}
\centering
  \caption{The best fitted models, on which the $\chi^{2}_r$ have value between 1 and 2. \label{tab:best-model12}}\
  \begin{tabular}{l l l l }
    \includegraphics[width=0.24\linewidth, clip]{Figs/model_120000_37.41_3.78.pdf} & \includegraphics[width=0.24\linewidth, clip]{Figs/model_160000_36.58_3.30.pdf} & \includegraphics[width=0.24\linewidth, clip]{Figs/model_120000_37.25_3.65.pdf} & \includegraphics[width=0.24\linewidth, clip]{Figs/model_130000_37.25_3.74.pdf} \\
    \includegraphics[width=0.24\linewidth, clip]{Figs/model_150000_37.12_3.70.pdf} & \includegraphics[width=0.24\linewidth, clip]{Figs/model_160000_36.93_3.60.pdf} & \includegraphics[width=0.24\linewidth, clip]{Figs/model_150000_37.19_3.78.pdf} & \includegraphics[width=0.24\linewidth, clip]{Figs/model_130000_37.27_3.74.pdf} \\
    \includegraphics[width=0.24\linewidth, clip]{Figs/model_140000_36.98_3.54.pdf} & \includegraphics[width=0.24\linewidth, clip]{Figs/model_120000_37.22_3.65.pdf} & \includegraphics[width=0.24\linewidth, clip]{Figs/model_130000_36.86_3.40.pdf} & \includegraphics[width=0.24\linewidth, clip]{Figs/model_130000_36.93_3.48.pdf} \\
    \includegraphics[width=0.24\linewidth, clip]{Figs/model_140000_37.12_3.65.pdf} & \includegraphics[width=0.24\linewidth, clip]{Figs/model_140000_37.22_3.74.pdf} & \includegraphics[width=0.24\linewidth, clip]{Figs/model_130000_37.15_3.65.pdf} & \includegraphics[width=0.24\linewidth, clip]{Figs/model_120000_37.35_3.74.pdf} \\
    \includegraphics[width=0.24\linewidth, clip]{Figs/model_140000_37.27_3.78.pdf} & \includegraphics[width=0.24\linewidth, clip]{Figs/model_160000_36.86_3.54.pdf} & \includegraphics[width=0.24\linewidth, clip]{Figs/model_120000_37.27_3.70.pdf} & \includegraphics[width=0.24\linewidth, clip]{Figs/model_130000_37.32_3.78.pdf} \\
    \includegraphics[width=0.24\linewidth, clip]{Figs/model_130000_37.03_3.54.pdf} & \includegraphics[width=0.24\linewidth, clip]{Figs/model_140000_37.08_3.65.pdf} & \includegraphics[width=0.24\linewidth, clip]{Figs/model_130000_37.22_3.70.pdf} & \includegraphics[width=0.24\linewidth, clip]{Figs/model_120000_37.32_3.74.pdf} \\
    \includegraphics[width=0.24\linewidth, clip]{Figs/model_160000_36.79_3.48.pdf} & \includegraphics[width=0.24\linewidth, clip]{Figs/model_200000_36.19_3.00.pdf} & \includegraphics[width=0.24\linewidth, clip]{Figs/model_130000_37.12_3.60.pdf} & \includegraphics[width=0.24\linewidth, clip]{Figs/model_150000_37.03_3.65.pdf} \\
    \includegraphics[width=0.24\linewidth, clip]{Figs/model_120000_37.39_3.78.pdf} & \includegraphics[width=0.24\linewidth, clip]{Figs/model_170000_36.58_3.30.pdf} & \includegraphics[width=0.24\linewidth, clip]{Figs/model_130000_37.08_3.60.pdf} & \includegraphics[width=0.24\linewidth, clip]{Figs/model_170000_36.43_3.18.pdf} \\
    \includegraphics[width=0.24\linewidth, clip]{Figs/model_150000_37.22_3.78.pdf} & \includegraphics[width=0.24\linewidth, clip]{Figs/model_170000_36.86_3.54.pdf} & \includegraphics[width=0.24\linewidth, clip]{Figs/model_160000_36.70_3.40.pdf} & \includegraphics[width=0.24\linewidth, clip]{Figs/model_120000_37.30_3.70.pdf} \\
    \includegraphics[width=0.24\linewidth, clip]{Figs/model_130000_37.30_3.78.pdf} & \includegraphics[width=0.24\linewidth, clip]{Figs/model_140000_37.03_3.60.pdf} & \includegraphics[width=0.24\linewidth, clip]{Figs/model_140000_36.70_3.30.pdf} & \includegraphics[width=0.24\linewidth, clip]{Figs/model_150000_36.93_3.54.pdf} \\
    \includegraphics[width=0.24\linewidth, clip]{Figs/model_140000_36.79_3.40.pdf} & \includegraphics[width=0.24\linewidth, clip]{Figs/model_140000_36.86_3.48.pdf} & \includegraphics[width=0.24\linewidth, clip]{Figs/model_140000_37.19_3.74.pdf} \\
  \end{tabular}
\end{table*}


%%%%%%%%%%%%%%%%%%%%%%%%%%%%%%%%%%%%%%%%%%%%%%%%%%


% Don't change these lines
\bsp	% typesetting comment
\label{lastpage}
\end{document}

% End of mnras_template.tex
